\documentclass{article} 
\usepackage{graphicx}
\usepackage{amsmath,amssymb,amsthm}
\usepackage{enumerate}
\usepackage{caption}
\usepackage{float}
\usepackage{hyperref}
\usepackage{tcolorbox}
\usepackage{mdframed}
\usepackage{changepage}
\usepackage[a4paper,margin=0.7in]{geometry}

\usepackage{fontspec}
% switch to a monospace font supporting more Unicode characters
\setmonofont{DejaVu Sans Mono}
\usepackage{minted}
\newmintinline[lean]{lean4}{bgcolor=white}
\newminted[leancode]{lean4}{fontsize=\footnotesize}

\usepackage{newunicodechar}

\newfontfamily{\freeserif}{DejaVu Sans}
\newunicodechar{✝}{\freeserif{✝}}
% Define Unicode symbols
\newunicodechar{ℕ}{\ensuremath{\mathbb{N}}}
\newunicodechar{ℝ}{\ensuremath{\mathbb{R}}}
\newunicodechar{→}{\ensuremath{\to}}
\newunicodechar{ᵐ}{\ensuremath{^{\mathrm{m}}}}
\newunicodechar{∀}{\ensuremath{\forall}}
\newunicodechar{↦}{\ensuremath{\mapsto}}
\newunicodechar{↔}{\ensuremath{\leftrightarrow}}
\newunicodechar{ₙ}{\ensuremath{_{\text{\freeserif{n}}}}}
\newunicodechar{⊔}{\ensuremath{\sqcup}}

\renewcommand{\texttt}[1]{{\small\ttfamily #1}}

\setlength{\parindent}{0pt}
\captionsetup{width=\linewidth}
\newtheorem{Theorem}{Theorem}

\begin{document}


\section*{Initial Thoughts and Statement}

The Lebesgue dominated convergence theorem (LDCT) is a classical result in measure theory has wide applications to many further topics. A possible statement of the theorem (given in the lecture notes for MATH50006 Measure Theory and Integration made by Dr. Pierre-François Rodriguez) is 

\begin{adjustwidth}{1cm}{1cm}
\begin{Theorem}[Lebesgue Dominated Convergence Theorem]
    Let \( g : X \to [0, \infty] \) be integrable, i.e., \( g \in L^1(\mu) \), and let \( f, f_n : X \to [-\infty, \infty] \), \( n \geq 1 \) be measurable functions such that
    \[
        |f_n(x)| \leq g(x) \quad \text{for all } x \in X, \quad \text{and} \quad f_n \xrightarrow{n \to \infty} f \quad \mu\text{-a.e.}
    \]
    Then,
    \[
        \int |f_n - f| \, d\mu 
        \xrightarrow{n \to \infty} 0.
    \]
\end{Theorem}
\end{adjustwidth}
\vspace{0.5cm}

To given a formal statement of this in lean, we need to first find ways to instantiate a measure space, represent an integral, impose the conditions of measurability and integrability and encode pointwise convergence \(\mu\) almost everywhere. The first statement I concocted of the theorem is

\begin{minted}[mathescape, numbersep=5pt, fontsize=\small]{Lean}
    variable {X : Type} [MeasurableSpace X] {μ : Measure X}
      {f : X → EReal} {fₙ : ℕ → X → EReal} {g : X → ENNReal}
    
    theorem LDCT
        (hg_int : Integrable g μ) -- integrability of g
        (hf_meas : Measurable f) -- measurability of f
        (hfn_meas : ∀ n, Measurable (fₙ n)) -- measurability of fₙ
        (hfn_to_f : ∀ᵐ (x : X) ∂μ, Tendsto (fun n ↦ fₙ n x) atTop (nhds (f x)))
        (hfn_bound : ∀ n, ∀ x, (fₙ n x).abs ≤ g x) -- dominated by g
        (hf_bound : ∀ x, (f x).abs ≤ g x) : -- f is dominated by g
        (Tendsto (fun n ↦ ∫⁻ x, (fₙ n x - f x).abs ∂μ) atTop (nhds 0))
\end{minted}

However, the error \texttt{failed to synthesize ENorm EReal} was raised. This error was raised because the extended reals does not have an in built norm. This was fixed by creating an instance of ENorm over EReal using \texttt{EReal.abs}. Furthermore, initially, I used \texttt{∫}, which is the Bochner integral which requires the functions being integrated to have a Banach space as its co-domain. However, as the integral in the output of the theorem is for \texttt{(fₙ n x - f x).abs} which is in \texttt{ENNReal}, we can use the lower Lebesgue integral implementation \texttt{∫⁻} for \texttt{NNReal} functions.

\section*{Strategy and Explanation of Formal Proof}

The informal proof given in the lecture notes is\\

\begin{adjustwidth}{1cm}{1cm}
\begin{proof}
We have $f_n \le g$ and $f_n \to f$ $\mu$-a.e.\ on $X$, hence $f_n, f \in L^1(\mu)$ by monotonicity of the lebesgue integral. Moreover,\( 
|f_n - f| \;\le\; |f_n| + |f| \;\le\; 2\,g,\)
thus $f_n - f \in L^1(\mu)$. Next, we apply Fatou's Lemma to the sequence of nonnegative functions \(2\,g - |f_n - f|\ge 0,\) which yields
\[
\int 2g \,d\mu
=\int \liminf_{n\to\infty}\; \bigl(2g - |f_n - f|\bigr)\,d\mu 
\le \liminf_{n\to\infty}\int \bigl(2g - |f_n - f|\bigr)\,d\mu = \int 2g \,d\mu + \liminf_{n\to\infty}\int - |f_n - f|\,d\mu
\]
Rearranging shows \(\limsup_{n\to\infty}\int |f_n - f|\,d\mu \le 0\)
and as the sequence is bounded below we have \( \lim_{n\to\infty}\int |f_n - f|\,d\mu = 0.\).
\end{proof}
\end{adjustwidth}

Looking at this proof we see there is one "big" calculation involved in showing that the limsup is bounded above by 0, which I expect to be the "meat" of the proof.

\subsection*{Integrability of $f$ and $f_n$ and the \texttt{bounded\_by\_integrable} lemma}

As the first two results we want are the integrability of $f$ and $f_n$, I decided to implement the following lemma

\begin{minted}[mathescape, numbersep=5pt, fontsize=\small]{Lean}
    lemma bounded_by_integrable (hf_meas : Measurable f) (hg_int : Integrable g μ)
    (hf_bound : ∀ x, (f x).abs ≤ g x) : Integrable f μ
\end{minted}

Informally, the proof for this would use the monoticity of the Lebesgue integral to show that \(\int |f|d\mu \le \int g \;d\mu <\infty\), implying \(f\) is integrable. Additionally, in Mathlib, to show that something is integrable it requires proofs of AEStrongMeasurablility and finite integral of norm. Measurability of a function is equivalent to AEStrongMeasurability if we are in a PseudoMetrizableSpace which EReal is. Therefore, we can extract the strong measurability using the theorem \texttt{aestronglyMeasurable} to get the first required proof. Next, we use monotonicity of the lower Lebesgue integral to get the finiteness of normed f. It is interesting to note that there is a lemma in Mathlib, \texttt{MeasureTheory.Integrable.mono'}, which gives this more directly but the fact that the co-domain of \(f\) is \texttt{EReals}, which is not a Banach space, seems to disqualify us from using this lemma. This would be the start of the issues with compatibility with \texttt{EReals}, \texttt{ENNReals} and reals. Originally, this was to be included in the statement of the LDCT as it is given in the lecture notes, but I have since removed it from the LDCT because it was not actually required by LEAN to close the main goal of LDCT. However, I included this lemma because I thought it could be useful and adds a new API.

\subsection*{Convergence of \(\int |f_n-f|d\mu\) to 0}

As in the informal proof, the main strategy would be to show that 0 bounds the liminf and limsup and thus the sequence of integrals must converge to 0. The lemma I found in Mathlib was \texttt{tendsto\_of\_le\_liminf\_of\_limsup\_le}. This lemma takes proofs that the sequence of integrals is bounded, liminf is bounded below and limsup is bounded above. Thus, I created 4 \texttt{have} statements and closed 3 pretty simply as follows:

\begin{enumerate}
    \item The sequence of integrals is bounded below by 0 and this was closed using \texttt{filter\_upwards} and \texttt{simp}.
    \item Likewise for the upper bound by \(\infty\).
    \item The lower bound of liminf by 0 is trivially closed by \texttt{simp}.
\end{enumerate}

Now, all that is left to show that limsup is bounded above by 0.

\subsection*{limsup bounded above by 0}

Initially, I thought I could have this within the proof itself but it quickly started running away from me so I made an auxilliary lemma, 


\begin{minted}[mathescape, numbersep=5pt, fontsize=\small]{Lean}
    lemma limsup_aux
    (hg_int : Integrable g μ) (hf_meas : Measurable f) (hfn_meas : ∀ n, Measurable (fₙ n))
    (hf_int : Integrable f μ) (hfn_int : ∀ n, Integrable (fₙ n) μ)
    (hfn_to_f : ∀ᵐ (x : X) ∂μ, Tendsto (fun n ↦ fₙ n x - f x) atTop (nhds 0))
    (hfn_bound : ∀ n, ∀ x, (fₙ n x).abs ≤ g x) (hf_bound : ∀ x, (f x).abs ≤ g x) :
    (limsup (fun n ↦ (∫⁻ x, (fₙ n x - f x).abs ∂μ)) atTop) ≤ 0 := by
\end{minted}

The main strategy in proving this is an application of Fatous's lemma for lim sup and this is given in Mathlib as \texttt{limsup\_lintegral\_le}. The calc block which does this is

\begin{minted}[mathescape, numbersep=5pt, fontsize=\small]{Lean}
    calc
        limsup (fun n ↦ (∫⁻ x, (fₙ n x - f x).abs ∂μ)) atTop
        ≤ ∫⁻ x, limsup (fun n ↦ (fₙ n x - f x).abs) atTop ∂μ :=
          @limsup_lintegral_le X _ μ (fun n ↦ (fun x ↦ (fₙ n x - f x).abs)) (2 * g)
            (h_meas) (hffn_le_2g) (h_2g_finite)
        _ = ∫⁻ x, 0 ∂μ := lintegral_congr_ae h_swap
        _ = 0 := by simp
\end{minted}

In this, the first inequality is an application of the mentioned Fatou's lemma. The second inequality is an application of \texttt{lintegral\_congr\_a} which says that the integral of functions which are equal almost everywhere are equal and the proof it eats, \texttt{h\_swap} was implemented in a have statement as below

\begin{minted}[mathescape, numbersep=5pt, fontsize=\small]{Lean}
    have h_swap : (fun x ↦ limsup (fun n ↦ (fₙ n x - f x).abs) atTop) =ᵐ[μ] 0 := by
        filter_upwards [hfn_to_f] with x hfn_to_f
        apply tendsto_EReal_abs at hfn_to_f
        apply Tendsto.limsup_eq at hfn_to_f
        simpa only [Pi.zero_apply]
\end{minted}

In this have, I made use of another lemma which should be in Mathlib which is that if a sequence in the EReal converges to 0 then the absolute value of 0, unfortunately, I could not prove this in time so I left it as a sorry. I got really close to it and the only issue was that I had the statement for convergence in the \texttt{EReal}s but I could not figure out how to use this to get convergence for the exact same sequence embedded in \texttt{ENNReal}s.

In pursuit of this proof, some other lemmas were implemented for \texttt{EReals}, namely

\begin{minted}[mathescape, numbersep=5pt, fontsize=\small]{Lean}

lemma abs_ne_top_iff {x : EReal} : x.abs ≠ ⊤ ↔ x ≠ ⊤ ∧ x ≠ ⊥

lemma EReal_abs_add_le_add {x y : EReal} : (x + y).abs ≤ x.abs + y.abs

lemma EReal_abs_sub_le_add (x y : EReal) : (x - y).abs ≤ x.abs + y.abs

lemma EReal_neg_abs_le (x : EReal) : -x ≤ x.abs

lemma EReal_le_abs_self (x : EReal) : x.abs ≤ -x ∨ x.abs ≤ x

lemma EReal_abs_le_max (x : EReal) : x.abs ≤ -x ⊔ x

lemma EReal_max_bounds_abs {a : ℕ → EReal} : 
    (fun n ↦ ↑(a n).abs) ≤ (fun n ↦ (a n) ⊔ -(a n))

lemma EReal_Tendsto_neg {a : ℕ → EReal} (h : Tendsto a atTop (nhds 0)) :
    Tendsto (fun n ↦ -a n) atTop (nhds 0)

lemma EReal_tendsto_max {a : ℕ → EReal} (h : Tendsto a atTop (nhds 0)) :
    Tendsto (fun n ↦ (a n) ⊔ -(a n)) atTop (nhds 0)

\end{minted}

Some useful techniques I used to prove these are induction over the \texttt{EReal}s so that I could deal with cases of top, bot and \(\mathbb{R}\) separately and the lift tactic.

\section*{Reflection}

Lastly, I set out to prove an additional corrolary but I did not reach this step. I however stated the theorem and introduced a Lesbesgue integral and some notation for it.\\

In this project I opened a can of worms. I found myself having to work with many translations from \texttt{EReal}s to \texttt{ENNReal}s and \(\mathbb{R}\) which was really hard. I understand why EReals do not have much API (because it is not really used anywhere) but by building some standard lemmas in Measure Theory and EReals, I gained familiarity with filters, induction on EReals, coercions and some of the lemmas for measure theory.

\end{document}